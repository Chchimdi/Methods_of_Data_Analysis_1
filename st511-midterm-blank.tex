% Options for packages loaded elsewhere
\PassOptionsToPackage{unicode}{hyperref}
\PassOptionsToPackage{hyphens}{url}
%
\documentclass[
]{article}
\title{ST 411/511 - Methods of Data Analysis I}
\usepackage{etoolbox}
\makeatletter
\providecommand{\subtitle}[1]{% add subtitle to \maketitle
  \apptocmd{\@title}{\par {\large #1 \par}}{}{}
}
\makeatother
\subtitle{Midterm Exam}
\author{Chimdi Chikezie}
\date{Due: Wednesday, February 9th before 11:59PM}

\usepackage{amsmath,amssymb}
\usepackage{lmodern}
\usepackage{iftex}
\ifPDFTeX
  \usepackage[T1]{fontenc}
  \usepackage[utf8]{inputenc}
  \usepackage{textcomp} % provide euro and other symbols
\else % if luatex or xetex
  \usepackage{unicode-math}
  \defaultfontfeatures{Scale=MatchLowercase}
  \defaultfontfeatures[\rmfamily]{Ligatures=TeX,Scale=1}
\fi
% Use upquote if available, for straight quotes in verbatim environments
\IfFileExists{upquote.sty}{\usepackage{upquote}}{}
\IfFileExists{microtype.sty}{% use microtype if available
  \usepackage[]{microtype}
  \UseMicrotypeSet[protrusion]{basicmath} % disable protrusion for tt fonts
}{}
\makeatletter
\@ifundefined{KOMAClassName}{% if non-KOMA class
  \IfFileExists{parskip.sty}{%
    \usepackage{parskip}
  }{% else
    \setlength{\parindent}{0pt}
    \setlength{\parskip}{6pt plus 2pt minus 1pt}}
}{% if KOMA class
  \KOMAoptions{parskip=half}}
\makeatother
\usepackage{xcolor}
\IfFileExists{xurl.sty}{\usepackage{xurl}}{} % add URL line breaks if available
\IfFileExists{bookmark.sty}{\usepackage{bookmark}}{\usepackage{hyperref}}
\hypersetup{
  pdftitle={ST 411/511 - Methods of Data Analysis I},
  pdfauthor={Chimdi Chikezie},
  hidelinks,
  pdfcreator={LaTeX via pandoc}}
\urlstyle{same} % disable monospaced font for URLs
\usepackage[margin=1in]{geometry}
\usepackage{color}
\usepackage{fancyvrb}
\newcommand{\VerbBar}{|}
\newcommand{\VERB}{\Verb[commandchars=\\\{\}]}
\DefineVerbatimEnvironment{Highlighting}{Verbatim}{commandchars=\\\{\}}
% Add ',fontsize=\small' for more characters per line
\usepackage{framed}
\definecolor{shadecolor}{RGB}{248,248,248}
\newenvironment{Shaded}{\begin{snugshade}}{\end{snugshade}}
\newcommand{\AlertTok}[1]{\textcolor[rgb]{0.94,0.16,0.16}{#1}}
\newcommand{\AnnotationTok}[1]{\textcolor[rgb]{0.56,0.35,0.01}{\textbf{\textit{#1}}}}
\newcommand{\AttributeTok}[1]{\textcolor[rgb]{0.77,0.63,0.00}{#1}}
\newcommand{\BaseNTok}[1]{\textcolor[rgb]{0.00,0.00,0.81}{#1}}
\newcommand{\BuiltInTok}[1]{#1}
\newcommand{\CharTok}[1]{\textcolor[rgb]{0.31,0.60,0.02}{#1}}
\newcommand{\CommentTok}[1]{\textcolor[rgb]{0.56,0.35,0.01}{\textit{#1}}}
\newcommand{\CommentVarTok}[1]{\textcolor[rgb]{0.56,0.35,0.01}{\textbf{\textit{#1}}}}
\newcommand{\ConstantTok}[1]{\textcolor[rgb]{0.00,0.00,0.00}{#1}}
\newcommand{\ControlFlowTok}[1]{\textcolor[rgb]{0.13,0.29,0.53}{\textbf{#1}}}
\newcommand{\DataTypeTok}[1]{\textcolor[rgb]{0.13,0.29,0.53}{#1}}
\newcommand{\DecValTok}[1]{\textcolor[rgb]{0.00,0.00,0.81}{#1}}
\newcommand{\DocumentationTok}[1]{\textcolor[rgb]{0.56,0.35,0.01}{\textbf{\textit{#1}}}}
\newcommand{\ErrorTok}[1]{\textcolor[rgb]{0.64,0.00,0.00}{\textbf{#1}}}
\newcommand{\ExtensionTok}[1]{#1}
\newcommand{\FloatTok}[1]{\textcolor[rgb]{0.00,0.00,0.81}{#1}}
\newcommand{\FunctionTok}[1]{\textcolor[rgb]{0.00,0.00,0.00}{#1}}
\newcommand{\ImportTok}[1]{#1}
\newcommand{\InformationTok}[1]{\textcolor[rgb]{0.56,0.35,0.01}{\textbf{\textit{#1}}}}
\newcommand{\KeywordTok}[1]{\textcolor[rgb]{0.13,0.29,0.53}{\textbf{#1}}}
\newcommand{\NormalTok}[1]{#1}
\newcommand{\OperatorTok}[1]{\textcolor[rgb]{0.81,0.36,0.00}{\textbf{#1}}}
\newcommand{\OtherTok}[1]{\textcolor[rgb]{0.56,0.35,0.01}{#1}}
\newcommand{\PreprocessorTok}[1]{\textcolor[rgb]{0.56,0.35,0.01}{\textit{#1}}}
\newcommand{\RegionMarkerTok}[1]{#1}
\newcommand{\SpecialCharTok}[1]{\textcolor[rgb]{0.00,0.00,0.00}{#1}}
\newcommand{\SpecialStringTok}[1]{\textcolor[rgb]{0.31,0.60,0.02}{#1}}
\newcommand{\StringTok}[1]{\textcolor[rgb]{0.31,0.60,0.02}{#1}}
\newcommand{\VariableTok}[1]{\textcolor[rgb]{0.00,0.00,0.00}{#1}}
\newcommand{\VerbatimStringTok}[1]{\textcolor[rgb]{0.31,0.60,0.02}{#1}}
\newcommand{\WarningTok}[1]{\textcolor[rgb]{0.56,0.35,0.01}{\textbf{\textit{#1}}}}
\usepackage{longtable,booktabs,array}
\usepackage{calc} % for calculating minipage widths
% Correct order of tables after \paragraph or \subparagraph
\usepackage{etoolbox}
\makeatletter
\patchcmd\longtable{\par}{\if@noskipsec\mbox{}\fi\par}{}{}
\makeatother
% Allow footnotes in longtable head/foot
\IfFileExists{footnotehyper.sty}{\usepackage{footnotehyper}}{\usepackage{footnote}}
\makesavenoteenv{longtable}
\usepackage{graphicx}
\makeatletter
\def\maxwidth{\ifdim\Gin@nat@width>\linewidth\linewidth\else\Gin@nat@width\fi}
\def\maxheight{\ifdim\Gin@nat@height>\textheight\textheight\else\Gin@nat@height\fi}
\makeatother
% Scale images if necessary, so that they will not overflow the page
% margins by default, and it is still possible to overwrite the defaults
% using explicit options in \includegraphics[width, height, ...]{}
\setkeys{Gin}{width=\maxwidth,height=\maxheight,keepaspectratio}
% Set default figure placement to htbp
\makeatletter
\def\fps@figure{htbp}
\makeatother
\setlength{\emergencystretch}{3em} % prevent overfull lines
\providecommand{\tightlist}{%
  \setlength{\itemsep}{0pt}\setlength{\parskip}{0pt}}
\setcounter{secnumdepth}{-\maxdimen} % remove section numbering
\ifLuaTeX
  \usepackage{selnolig}  % disable illegal ligatures
\fi

\begin{document}
\maketitle

\hypertarget{directions}{%
\subsubsection{Directions:}\label{directions}}

Please read the directions below \emph{very} carefully as failure to
follow them will result in point deductions, receiving a zero on the
exam, and/or having your name forwarded to the Office of Student
Affairs.

\begin{itemize}
\tightlist
\item
  Submissions must be made as entirely typed PDFs to Gradescope prior to
  the deadline. Exams which are handwritten or contain images, scans,
  etc. of handwritten work will not be accepted/graded.
\item
  No late submissions are allowed for any reason unless you have prior
  approval from the instructor. See the course syllabus for more
  details.

  \begin{itemize}
  \tightlist
  \item
    Failing to ``knit'' your document to PDF is not a valid reason to
    submit your exam after the deadline. Start the exam early and knit
    your document frequently.
  \item
    Failing to upload your document to Gradescope is not a valid reason
    to submit your exam after the deadline. Verify your exam has been
    successfully submitted and allow yourself time to ``troubleshoot''
    any potential issues.
  \item
    Failing to upload the correct document to Gradescope is not a valid
    reason to submit your exam after the deadline. Verify that you've
    uploaded the correct document prior to the deadline.
  \end{itemize}
\item
  Exams should be ``knitted'' directly to PDF using the provided
  template.

  \begin{itemize}
  \tightlist
  \item
    Knitting your document to a Word Doc file and then converting it to
    PDF is not desirable but I'll live if you do this.
  \item
    Do not knit your document to an HTML file and then convert it to
    PDF. The formatting is terrible and it makes them difficult to read.
    Exams where this occurs will be penalized.
  \end{itemize}
\item
  Indicate all pages containing solutions to the individual questions in
  Gradescope. Exams which are not properly indicated will be penalized
  on a per-question basis.
\item
  Do not include extraneous code/outputs (i.e.~``Scratch work''). Exams
  which include extraneous code/outputs will be penalized.
\item
  Answer questions using complete sentences. Solutions which are not
  written as complete sentences or are difficult to read and/or
  understand will be penalized.
\item
  Exams which are messy, hard to read, unclear, disorganized, etc. will
  be penalized.
\item
  The exam is open book and open note. You can utilize any material
  found in the text or the course Canvas page but nothing else.
\item
  Do not talk or seek assistance from anyone else (students, peers,
  other faculty, online message boards, etc.).

  \begin{itemize}
  \tightlist
  \item
    Exams where ``copying'' or student collaboration is suspected will
    receive grades of zero until students arrange to meet with the
    instructor to explain the similarities.
  \end{itemize}
\item
  Do not share code or solutions with anyone else.
\end{itemize}

\newpage

\hypertarget{question-1-10-points}{%
\subsection{Question 1 (10 points)}\label{question-1-10-points}}

For each part below, indicate whether the statement is TRUE or FALSE and
provide a 1-2 sentence explanation as to why.

\hypertarget{a.-2-points-if-a-95-confidence-interval-for-an-unknown-population-mean-is-21.4-78.7-this-means-there-is-a-95-probability-that-the-true-population-mean-is-between-21.4-and-78.7.}{%
\subsubsection{\texorpdfstring{a. (2 points) If a 95\% confidence
interval for an unknown population mean is (21.4, 78.7), this means
there is a 95\% \emph{probability} that the true population mean is
between 21.4 and
78.7.}{a. (2 points) If a 95\% confidence interval for an unknown population mean is (21.4, 78.7), this means there is a 95\% probability that the true population mean is between 21.4 and 78.7.}}\label{a.-2-points-if-a-95-confidence-interval-for-an-unknown-population-mean-is-21.4-78.7-this-means-there-is-a-95-probability-that-the-true-population-mean-is-between-21.4-and-78.7.}}

\begin{itemize}
\tightlist
\item
  Answer: FALSE
\item
  Reason: Confidence intervals is not probability. It gives us 95\%
  confidence that the true population mean will fall between 21.4 and
  78.7.
\end{itemize}

\hypertarget{b.-2-points-after-conducting-a-z-test-the-corresponding-p-value-of-the-test-is-0.0456.-this-means-precisely-that-the-probability-that-the-null-hypothesis-is-true-is-0.0456.}{%
\subsubsection{b. (2 points) After conducting a Z-test, the
corresponding p-value of the test is 0.0456. This means, precisely, that
the probability that the null hypothesis is true is
0.0456.}\label{b.-2-points-after-conducting-a-z-test-the-corresponding-p-value-of-the-test-is-0.0456.-this-means-precisely-that-the-probability-that-the-null-hypothesis-is-true-is-0.0456.}}

\begin{itemize}
\tightlist
\item
  Answer: FALSE
\item
  Reason: P-value does not determine if the null hypothesis is true,
  however, it determines the probability of getting a result at least as
  extreme as the statistic observed..
\end{itemize}

\hypertarget{c.-2-points-in-a-study-on-handedness-in-households-with-exactly-two-children-it-was-found-that-older-siblings-are-much-more-likely-to-be-left-handed-than-younger-siblings-a-statistically-significant-result.-the-researchers-should-conclude-that-being-born-first-causes-the-older-children-at-least-in-some-cases-to-be-left-handed.}{%
\subsubsection{\texorpdfstring{c.~(2 points) In a study on
``handedness'' in households with exactly two children, it was found
that older siblings are much more likely to be left-handed than younger
siblings (A statistically significant result). The researchers should
conclude that being born first \emph{causes} the older children (at
least in some cases) to be
left-handed.}{c.~(2 points) In a study on ``handedness'' in households with exactly two children, it was found that older siblings are much more likely to be left-handed than younger siblings (A statistically significant result). The researchers should conclude that being born first causes the older children (at least in some cases) to be left-handed.}}\label{c.-2-points-in-a-study-on-handedness-in-households-with-exactly-two-children-it-was-found-that-older-siblings-are-much-more-likely-to-be-left-handed-than-younger-siblings-a-statistically-significant-result.-the-researchers-should-conclude-that-being-born-first-causes-the-older-children-at-least-in-some-cases-to-be-left-handed.}}

\begin{itemize}
\tightlist
\item
  Answer: FALSE
\item
  Reason: This is an observational study and we cannot make a causal
  conclusion. There might be other factors that make those children to
  be left handed and not only being a first born.
\end{itemize}

\hypertarget{d.-2-points-the-los-angeles-dodgers-a-major-league-baseball-team-played-81-home-games-against-17-other-teams.-for-these-games-the-dodgers-played-the-same-team-over-the-course-of-two-to-four-consecutive-nights.-since-these-were-different-games-i.e.-they-always-started-with-a-score-of-0-0-the-games-should-be-considered-to-be-independent-of-one-another.}{%
\subsubsection{\texorpdfstring{d.~(2 points) The Los Angeles Dodgers, a
major league baseball team, played 81 home games against 17 other teams.
For these games, the Dodgers played the same team over the course of two
to four consecutive nights. Since these were different games, i.e.~they
always started with a score of 0-0, the games should be considered to be
\emph{independent} of one
another.}{d.~(2 points) The Los Angeles Dodgers, a major league baseball team, played 81 home games against 17 other teams. For these games, the Dodgers played the same team over the course of two to four consecutive nights. Since these were different games, i.e.~they always started with a score of 0-0, the games should be considered to be independent of one another.}}\label{d.-2-points-the-los-angeles-dodgers-a-major-league-baseball-team-played-81-home-games-against-17-other-teams.-for-these-games-the-dodgers-played-the-same-team-over-the-course-of-two-to-four-consecutive-nights.-since-these-were-different-games-i.e.-they-always-started-with-a-score-of-0-0-the-games-should-be-considered-to-be-independent-of-one-another.}}

\begin{itemize}
\tightlist
\item
  Answer: TRUE
\item
  Reason: Because the winning probability of all the games are the same
  for all games.
\end{itemize}

\hypertarget{e.-2-points-a-professional-dk64-speedrunner-and-current-101-record-holder-wishes-to-purchase-14-different-candies-for-their-significant-other-on-valentines-day.-they-visit-two-different-large-grocery-stores-and-record-the-prices-of-the-same-14-candies.-if-they-want-to-see-if-the-average-price-of-the-14-candies-are-different-between-the-two-stores-then-they-should-conduct-an-equal-variance-or-a-welchs-two-sample-t-test.}{%
\subsubsection{e. (2 points) A professional DK64 speedrunner, and
current 101\% record holder, wishes to purchase 14 different candies for
their significant other on Valentine's Day. They visit two different
large grocery stores and record the prices of the same 14 candies. If
they want to see if the average price of the 14 candies are different
between the two stores then they should conduct an equal variance or a
Welch's two-sample
t-test.}\label{e.-2-points-a-professional-dk64-speedrunner-and-current-101-record-holder-wishes-to-purchase-14-different-candies-for-their-significant-other-on-valentines-day.-they-visit-two-different-large-grocery-stores-and-record-the-prices-of-the-same-14-candies.-if-they-want-to-see-if-the-average-price-of-the-14-candies-are-different-between-the-two-stores-then-they-should-conduct-an-equal-variance-or-a-welchs-two-sample-t-test.}}

\begin{itemize}
\tightlist
\item
  Answer: Welch's two-sample t-test
\item
  Reason: Welch's two-sample t-test is used to test that means are the
  same when their variances are significantly different.
\end{itemize}

\newpage

\hypertarget{question-2-13-points}{%
\subsection{Question 2 (13 points)}\label{question-2-13-points}}

Dairy scientists are interested in researching the ascorbic acid content
of dairy cows raised in Benton County. Specifically, they would like to
know if the population mean ascorbic acid content (measured in mg. per
cc), is different than 220 mg/cc. To conduct their study, they gather a
random sample from six different cows raised in Benton County, the
values or which are below (You can \emph{knit} the exam to see the
printed table):

\begin{longtable}[]{@{}rrrrrr@{}}
\toprule
\endhead
255 & 190 & 55 & 137 & 138 & 174 \\
\bottomrule
\end{longtable}

\hypertarget{a.-2-points-based-on-the-researchers-question-of-interest-write-out-the-appropriate-null-and-alternative-hypotheses-for-a-statistical-hypothesis-test-using-correct-statistical-notation.}{%
\subsubsection{a. (2 points) Based on the researcher's question of
interest, write out the appropriate null and alternative hypotheses for
a statistical hypothesis test using correct statistical
notation.}\label{a.-2-points-based-on-the-researchers-question-of-interest-write-out-the-appropriate-null-and-alternative-hypotheses-for-a-statistical-hypothesis-test-using-correct-statistical-notation.}}

\begin{align*}
H_0:&\mu_p =220 mg/cc\\

H_A:&\mu_P \neq 220 mg/cc
\end{align*}

\hypertarget{b.-4-points-conduct-an-appropriate-statistical-hypothesis-test-by-hand-i.e.-you-can-use-r-code-or-mathematical-notation-but-do-not-use-any-built-in-test-functions-like-t.test-wilcox.test-binom.test-etc.-to-answer-the-researchers-question-of-interest-at-the-5-significance-level.-specifically-write-out-the-value-of-the-test-statistic-the-p-value-and-a-conclusion-in-the-spaces-provided-below.-for-the-conclusion-be-sure-to-state-the-statistical-outcome-of-the-test-why-you-arrived-at-that-conclusion-and-what-this-conclusion-means-for-the-researchers-using-the-context-of-the-question.-please-show-all-relevant-work-in-the-space-indicatedprovided-below.}{%
\subsubsection{\texorpdfstring{b. (4 points) Conduct an appropriate
statistical hypothesis test, by hand (i.e.~you can use R code or
mathematical notation, but do not use any built-in test functions like
\texttt{t.test()}, \texttt{wilcox.test()}, \texttt{binom.test()}, etc.),
to answer the researcher's question of interest at the 5\% significance
level. Specifically, write out the value of the test statistic, the
p-value, and a conclusion in the spaces provided below. For the
conclusion, be sure to state the statistical outcome of the test, why
you arrived at that conclusion, and what this conclusion means for the
researchers \textbf{using the context of the question}. Please show all
relevant work in the space indicated/provided
below.}{b. (4 points) Conduct an appropriate statistical hypothesis test, by hand (i.e.~you can use R code or mathematical notation, but do not use any built-in test functions like t.test(), wilcox.test(), binom.test(), etc.), to answer the researcher's question of interest at the 5\% significance level. Specifically, write out the value of the test statistic, the p-value, and a conclusion in the spaces provided below. For the conclusion, be sure to state the statistical outcome of the test, why you arrived at that conclusion, and what this conclusion means for the researchers using the context of the question. Please show all relevant work in the space indicated/provided below.}}\label{b.-4-points-conduct-an-appropriate-statistical-hypothesis-test-by-hand-i.e.-you-can-use-r-code-or-mathematical-notation-but-do-not-use-any-built-in-test-functions-like-t.test-wilcox.test-binom.test-etc.-to-answer-the-researchers-question-of-interest-at-the-5-significance-level.-specifically-write-out-the-value-of-the-test-statistic-the-p-value-and-a-conclusion-in-the-spaces-provided-below.-for-the-conclusion-be-sure-to-state-the-statistical-outcome-of-the-test-why-you-arrived-at-that-conclusion-and-what-this-conclusion-means-for-the-researchers-using-the-context-of-the-question.-please-show-all-relevant-work-in-the-space-indicatedprovided-below.}}

\begin{itemize}
\item
  Test statistic: (xbar - mu) / sqrt(s/n) = (158.167 - 220) /
  sqrt(4427.767/6) = -2.276164
\item
  p-value: 2 * (1 - pt(abs(150.0685), 6 - 1)) = 0.071882
\item
  Conclusion: I fail to reject the null hypothesis because the P-value
  is greater than alpha. We do have have enough evidence to show that
  the population mean ascorbic acid content (measured in mg. per cc), is
  equal to 220 mg/cc.
\end{itemize}

\textbf{Show your work for this question below. If you write R code,
please make sure the computed test statistic and p-value are shown as
outputs (And please don't leave a bunch of extraneous code/outputs in
your solutions).}

\begin{Shaded}
\begin{Highlighting}[]
\NormalTok{val }\OtherTok{\textless{}{-}} \FunctionTok{c}\NormalTok{(}\DecValTok{255}\NormalTok{, }\DecValTok{190}\NormalTok{, }\DecValTok{55}\NormalTok{, }\DecValTok{137}\NormalTok{, }\DecValTok{138}\NormalTok{, }\DecValTok{174}\NormalTok{)}

\NormalTok{mu }\OtherTok{\textless{}{-}} \DecValTok{220}
\NormalTok{xbar }\OtherTok{\textless{}{-}} \FloatTok{158.167}
\NormalTok{s }\OtherTok{\textless{}{-}} \FloatTok{4427.767}
\NormalTok{n }\OtherTok{\textless{}{-}} \DecValTok{6}

\NormalTok{t }\OtherTok{\textless{}{-}}\NormalTok{ (xbar }\SpecialCharTok{{-}}\NormalTok{ mu) }\SpecialCharTok{/} \FunctionTok{sqrt}\NormalTok{(s}\SpecialCharTok{/}\NormalTok{n)}
\FunctionTok{sprintf}\NormalTok{(}\StringTok{"t = \%f"}\NormalTok{, t)}
\end{Highlighting}
\end{Shaded}

\begin{verbatim}
## [1] "t = -2.276164"
\end{verbatim}

\begin{Shaded}
\begin{Highlighting}[]
\NormalTok{pvalue }\OtherTok{\textless{}{-}} \DecValTok{2} \SpecialCharTok{*}\NormalTok{ (}\DecValTok{1} \SpecialCharTok{{-}} \FunctionTok{pt}\NormalTok{(}\FunctionTok{abs}\NormalTok{(t), n }\SpecialCharTok{{-}} \DecValTok{1}\NormalTok{))}
\FunctionTok{sprintf}\NormalTok{(}\StringTok{"pvalue = \%f"}\NormalTok{, pvalue)}
\end{Highlighting}
\end{Shaded}

\begin{verbatim}
## [1] "pvalue = 0.071882"
\end{verbatim}

\hypertarget{c.-3-points-compute-by-hand-and-interpret-a-95-confidence-interval-for-the-population-mean.-write-your-confidence-interval-in-interval-form-i.e.-lower-bound-upperbound-and-interpret-the-interval-in-the-context-of-this-particular-question.-show-your-work-in-the-space-indicated-below.}{%
\subsubsection{\texorpdfstring{c.~(3 points) Compute, by hand, and
interpret a 95\% confidence interval for the population mean. Write your
confidence interval in ``interval form'', i.e.~\emph{(lower bound,
upperbound)}, and interpret the interval \textbf{in the context of this
particular question}. Show your work in the space indicated
below.}{c.~(3 points) Compute, by hand, and interpret a 95\% confidence interval for the population mean. Write your confidence interval in ``interval form'', i.e.~(lower bound, upperbound), and interpret the interval in the context of this particular question. Show your work in the space indicated below.}}\label{c.-3-points-compute-by-hand-and-interpret-a-95-confidence-interval-for-the-population-mean.-write-your-confidence-interval-in-interval-form-i.e.-lower-bound-upperbound-and-interpret-the-interval-in-the-context-of-this-particular-question.-show-your-work-in-the-space-indicated-below.}}

\begin{itemize}
\tightlist
\item
  95\% confidence interval: (88.33601, 227.998)
\end{itemize}

CI upper\_limit \textless- x +- (error * sqrt(s/n))

error = 1 - (0.05/2), 6 - 1 = 0.975,5 qt(0.975,5) = 2.570582

CI lower bound = 158.166667 - (2.571 * sqrt(4427.767/6)) = 88.33601 CI
upper bound = 158.166667 + (2.571 * sqrt(4427.767/6)) = 227.998

\begin{itemize}
\tightlist
\item
  Interpretation:
\end{itemize}

We are 95\% confident that the true population mean ascorbic acid
content (measured in mg. per cc) of dairy cows raised in Benton County
is between 88.33601 and 227.998

\textbf{Show your work for this question below. If you write R code,
please make sure the computed values are shown as outputs (And please
don't leave a bunch of extraneous code/outputs in your solutions).}

\begin{Shaded}
\begin{Highlighting}[]
\NormalTok{cv }\OtherTok{\textless{}{-}} \DecValTok{1} \SpecialCharTok{{-}}\NormalTok{ (}\FloatTok{0.05}\SpecialCharTok{/}\DecValTok{2}\NormalTok{)}
\NormalTok{error }\OtherTok{\textless{}{-}} \FunctionTok{qt}\NormalTok{(cv,}\DecValTok{5}\NormalTok{)}

\NormalTok{lowerbound }\OtherTok{\textless{}{-}}\NormalTok{ xbar }\SpecialCharTok{{-}}\NormalTok{ (error }\SpecialCharTok{*} \FunctionTok{sqrt}\NormalTok{(s}\SpecialCharTok{/}\NormalTok{n))}
\FunctionTok{sprintf}\NormalTok{(}\StringTok{"lowerbound = \%f"}\NormalTok{, lowerbound)}
\end{Highlighting}
\end{Shaded}

\begin{verbatim}
## [1] "lowerbound = 88.336012"
\end{verbatim}

\begin{Shaded}
\begin{Highlighting}[]
\NormalTok{upperbound }\OtherTok{\textless{}{-}}\NormalTok{ xbar }\SpecialCharTok{+}\NormalTok{ (error }\SpecialCharTok{*} \FunctionTok{sqrt}\NormalTok{(s}\SpecialCharTok{/}\NormalTok{n))}
\FunctionTok{sprintf}\NormalTok{(}\StringTok{"upperbound = \%f"}\NormalTok{, upperbound)}
\end{Highlighting}
\end{Shaded}

\begin{verbatim}
## [1] "upperbound = 227.997988"
\end{verbatim}

\hypertarget{d.-2-points-in-discussing-the-outcome-of-your-hypothesis-test-in-part-b.-with-the-researchers-they-inform-you-that-they-believe-that-the-population-distribution-is-likely-to-be-heavily-and-i-mean-heavily-right-skewed.-why-does-this-new-knowledge-about-the-population-distribution-make-you-doubt-the-validity-of-the-test-you-conducted-in-part-b.-be-specific-and-be-sure-to-explain-the-underlying-reasoning-in-your-explanation.}{%
\subsubsection{\texorpdfstring{d.~(2 points) In discussing the outcome
of your hypothesis test in part b. with the researchers, they inform you
that they believe that the population distribution is likely to be
heavily (and I mean \emph{heavily}) right-skewed. Why does this new
knowledge about the population distribution make you doubt the validity
of the test you conducted in part b.? Be specific and be sure to explain
the underlying reasoning in your
explanation.}{d.~(2 points) In discussing the outcome of your hypothesis test in part b. with the researchers, they inform you that they believe that the population distribution is likely to be heavily (and I mean heavily) right-skewed. Why does this new knowledge about the population distribution make you doubt the validity of the test you conducted in part b.? Be specific and be sure to explain the underlying reasoning in your explanation.}}\label{d.-2-points-in-discussing-the-outcome-of-your-hypothesis-test-in-part-b.-with-the-researchers-they-inform-you-that-they-believe-that-the-population-distribution-is-likely-to-be-heavily-and-i-mean-heavily-right-skewed.-why-does-this-new-knowledge-about-the-population-distribution-make-you-doubt-the-validity-of-the-test-you-conducted-in-part-b.-be-specific-and-be-sure-to-explain-the-underlying-reasoning-in-your-explanation.}}

If we have an outlier, it can make the distribution rightly skewed, then
the mean value will be changed based on those outliers and we will not
be able to get a good estimation of our distribution from the mean the
value. T-test depends on mean value and when the mean value is not good
enough, it looses its validity. This is the reason that makes me doubt
the validity of the test you conducted in part b.

\hypertarget{e.-2-points-is-this-study-an-example-of-a-randomized-experiment-or-an-observational-study-how-can-you-tell}{%
\subsubsection{e. (2 points) Is this study an example of a randomized
experiment or an observational study? How can you
tell?}\label{e.-2-points-is-this-study-an-example-of-a-randomized-experiment-or-an-observational-study-how-can-you-tell}}

It is an observational study because the researcher is not influencing
the ascorbic acid content.

\newpage

\hypertarget{question-3-12-points}{%
\subsection{Question 3 (12 points)}\label{question-3-12-points}}

A European professional basketball league is trying to decide whether a
new workout supplement should be banned for also being a ``performance
enhancing'' drug. To test the workout supplement, they measure how many
free-throws a person successfully makes while shooting a basketball
through a standard, regulation height hoop in 4 minutes. For each of the
11 randomly selected individuals, the league measures the number of
successful free-throws the person makes while \emph{unmedicated} (before
taking the new supplement) and again while \emph{medicated} (after
taking the drug). If more free throws are made while \emph{medicated}
than \emph{unmedicated} then the league will ban the supplement. A
variety of summaries are given below (Note: ``(Medicated -
Unmedicated)'' denotes the pairwise differences for each observation):

\begin{longtable}[]{@{}cccc@{}}
\toprule
& Unmedicated & Medicated & (Medicated - Unmedicated) \\
\midrule
\endhead
\(n\) & 11 & 11 & 11 \\
\(\bar{X}\) & 20.73 & 23.55 & 2.82 \\
\(s^2\) & 21.42 & 29.27 & 10.56 \\
\bottomrule
\end{longtable}

\hypertarget{a.-2-points-using-correct-statistical-notation-write-out-the-null-and-alternative-hypotheses-needed-to-conduct-a-t-test-which-will-help-the-league-decide-whether-or-not-they-should-ban-the-supplement.}{%
\subsubsection{a. (2 points) Using correct statistical notation, write
out the null and alternative hypotheses needed to conduct a t-test which
will help the league decide whether or not they should ban the
supplement.}\label{a.-2-points-using-correct-statistical-notation-write-out-the-null-and-alternative-hypotheses-needed-to-conduct-a-t-test-which-will-help-the-league-decide-whether-or-not-they-should-ban-the-supplement.}}

\begin{align*}
H_0:&\mu_M - \mu_U=0\\
H_A:&\mu_M - \mu_U> 0
\end{align*}

\hypertarget{the-r-output-below-shows-some-of-the-output-from-two-different-t-tests-one-of-which-is-the-equal-variance-independent-two-sample-t-test-and-the-other-is-for-a-paired-t-test.}{%
\paragraph{The R output below shows some of the output from two
different t-tests, one of which is the equal variance (independent)
two-sample t-test and the other is for a paired
t-test.}\label{the-r-output-below-shows-some-of-the-output-from-two-different-t-tests-one-of-which-is-the-equal-variance-independent-two-sample-t-test-and-the-other-is-for-a-paired-t-test.}}

\begin{verbatim}
    Two Sample t-test

data:  Medicated and Unmedicated
t = 1.3128, df = 20, p-value = 0.1021
sample estimates:
mean of Medicated mean of Unmedicated 
         23.54545            20.72727 
\end{verbatim}

\begin{verbatim}
    Paired t-test

data:  Medicated and Unmedicated
t = 2.8758, df = 10, p-value = 0.008252
sample estimates:
mean of the differences 
               2.818182 
\end{verbatim}

\hypertarget{b.-2-points-which-one-of-the-above-tests-is-the-most-appropriate-to-use-to-answer-the-question-of-interest-explain-why}{%
\subsubsection{b. (2 points) Which one of the above tests is the most
appropriate to use to answer the question of interest? Explain
why?}\label{b.-2-points-which-one-of-the-above-tests-is-the-most-appropriate-to-use-to-answer-the-question-of-interest-explain-why}}

Paired t-test. This is because the European professional basketball
league is created 2 pairs: 1. Before testing 2. After testing

\hypertarget{c.-2-points-using-the-test-output-you-chose-in-part-b.-and-the-hypotheses-you-stated-in-part-a.-write-a-conclusion-for-the-test.-let-alpha-0.05.-be-sure-to-1-write-a-sentence-in-regards-to-the-outcome-of-your-test-2-a-brief-explanation-as-to-why-you-chose-that-particular-outcome-and-3-what-the-outcome-of-your-test-means-in-the-context-of-the-problem.}{%
\subsubsection{\texorpdfstring{c.~(2 points) Using the test output you
chose in part b. and the hypotheses you stated in part a., write a
conclusion for the test. Let \(\alpha = 0.05\). Be sure to (1) write a
sentence in regards to the outcome of your test, (2) a brief explanation
as to why you chose that particular outcome, and (3) what the outcome of
your test means in the context of the
problem.}{c.~(2 points) Using the test output you chose in part b. and the hypotheses you stated in part a., write a conclusion for the test. Let \textbackslash alpha = 0.05. Be sure to (1) write a sentence in regards to the outcome of your test, (2) a brief explanation as to why you chose that particular outcome, and (3) what the outcome of your test means in the context of the problem.}}\label{c.-2-points-using-the-test-output-you-chose-in-part-b.-and-the-hypotheses-you-stated-in-part-a.-write-a-conclusion-for-the-test.-let-alpha-0.05.-be-sure-to-1-write-a-sentence-in-regards-to-the-outcome-of-your-test-2-a-brief-explanation-as-to-why-you-chose-that-particular-outcome-and-3-what-the-outcome-of-your-test-means-in-the-context-of-the-problem.}}

We reject the null hypothesis at 0.05 significance level because our
p-value is less than alpha. We have evidence that more free throws were
made while medicated than unmedicated, therefore the league will ban the
supplement.

\hypertarget{d.-2-points-which-of-the-following-plots-best-represents-the-p-value-associated-with-this-particular-hypothesis-test-here-assume-the-shaded-region-represents-the-p-value-of-the-test.}{%
\subsubsection{\texorpdfstring{d.~(2 points) Which of the following
plots best represents the \(p\)-value associated with this particular
hypothesis test? Here, assume the shaded region represents the
\(p\)-value of the
test.}{d.~(2 points) Which of the following plots best represents the p-value associated with this particular hypothesis test? Here, assume the shaded region represents the p-value of the test.}}\label{d.-2-points-which-of-the-following-plots-best-represents-the-p-value-associated-with-this-particular-hypothesis-test-here-assume-the-shaded-region-represents-the-p-value-of-the-test.}}

\begin{center}\includegraphics{st511-midterm-blank_files/figure-latex/unnamed-chunk-3-1} \end{center}

The p-value is best represented by plot: (C)

\hypertarget{e.-2-points-based-on-the-test-you-chose-in-part-b.-compute-by-hand-a-95-confidence-interval.-write-your-confidence-interval-in-interval-form-i.e.-lower-bound-upperbound-and-show-your-work-in-the-space-indicated-below.-note-you-do-not-need-to-interpret-this-confidence-interval.}{%
\subsubsection{\texorpdfstring{e. (2 points) Based on the test you chose
in part b., compute, by hand, a 95\% confidence interval. Write your
confidence interval in ``interval form'', i.e.~\emph{(lower bound,
upperbound)} and show your work in the space indicated below. Note: You
do not need to interpret this confidence
interval.}{e. (2 points) Based on the test you chose in part b., compute, by hand, a 95\% confidence interval. Write your confidence interval in ``interval form'', i.e.~(lower bound, upperbound) and show your work in the space indicated below. Note: You do not need to interpret this confidence interval.}}\label{e.-2-points-based-on-the-test-you-chose-in-part-b.-compute-by-hand-a-95-confidence-interval.-write-your-confidence-interval-in-interval-form-i.e.-lower-bound-upperbound-and-show-your-work-in-the-space-indicated-below.-note-you-do-not-need-to-interpret-this-confidence-interval.}}

\begin{itemize}
\tightlist
\item
  95\% confidence interval: (1.042340, 4.594024)
\end{itemize}

\textbf{Show your work for this question below. If you write R code,
please make sure the computed values are shown as outputs (And please
don't leave a bunch of extraneous code/outputs in your solutions).}

\begin{Shaded}
\begin{Highlighting}[]
\NormalTok{cv }\OtherTok{\textless{}{-}} \DecValTok{1} \SpecialCharTok{{-}}\NormalTok{ (}\FloatTok{0.05}\NormalTok{)}
\NormalTok{error }\OtherTok{\textless{}{-}} \FunctionTok{qt}\NormalTok{(cv,}\DecValTok{10}\NormalTok{)}
\NormalTok{s }\OtherTok{\textless{}{-}} \FloatTok{10.56} 
\NormalTok{xbar }\OtherTok{\textless{}{-}} \FloatTok{2.818182}
\NormalTok{n }\OtherTok{\textless{}{-}} \DecValTok{11}

\NormalTok{lowerbound }\OtherTok{\textless{}{-}}\NormalTok{ xbar }\SpecialCharTok{{-}}\NormalTok{ (error }\SpecialCharTok{*} \FunctionTok{sqrt}\NormalTok{(s}\SpecialCharTok{/}\NormalTok{n))}
\FunctionTok{sprintf}\NormalTok{(}\StringTok{"lowerbound = \%f"}\NormalTok{, lowerbound)}
\end{Highlighting}
\end{Shaded}

\begin{verbatim}
## [1] "lowerbound = 1.042340"
\end{verbatim}

\begin{Shaded}
\begin{Highlighting}[]
\NormalTok{upperbound }\OtherTok{\textless{}{-}}\NormalTok{ xbar }\SpecialCharTok{+}\NormalTok{ (error }\SpecialCharTok{*} \FunctionTok{sqrt}\NormalTok{(s}\SpecialCharTok{/}\NormalTok{n))}
\FunctionTok{sprintf}\NormalTok{(}\StringTok{"upperbound = \%f"}\NormalTok{, upperbound)}
\end{Highlighting}
\end{Shaded}

\begin{verbatim}
## [1] "upperbound = 4.594024"
\end{verbatim}

\hypertarget{f.-2-points-explain-why-we-fail-to-reject-rather-than-accept-a-null-hypothesis-when-our-test-statistic-is-less-extreme-than-the-critical-value.}{%
\subsubsection{f.~(2 points) Explain why we ``fail to reject'' rather
than ``accept'' a null hypothesis when our test statistic is less
extreme than the critical
value.}\label{f.-2-points-explain-why-we-fail-to-reject-rather-than-accept-a-null-hypothesis-when-our-test-statistic-is-less-extreme-than-the-critical-value.}}

Accepting the null hypothesis would indicate that we have proven an
effect doesn't exist. Failing to reject the null indicates that our
sample did not provide sufficient evidence to conclude that the effect
exists.

\end{document}
